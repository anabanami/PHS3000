
% latexmk -pvc -pdf
\documentclass[12pt,a4paper]{article}
\usepackage{amsmath,amsthm,amsfonts,amssymb}
\usepackage[english]{babel}
\usepackage{blindtext}
\usepackage[margin=0.5in]{geometry}

\title{Measurement of $\beta-$ray spectra}
\author{Ana C. Fabela Hinojosa\\
\small{School of Physics and Astronomy, Monash University}}
\date{Tuesday 18\textsuperscript{th} August, 2020}
\begin{document}
\maketitle
\begin{abstract}
Using a thin lens magnetic spectrometer, we measure the momentum spectrum of electrons emitted as $\beta^{-}$ particles from a radioactive source of  \textsuperscript{137}Cs. 
The momentum of the radiated electrons is defined by the adjustable magnetic lens current and $k$ a proportionality constant. The spectrometer's lens current modulates the magnetic field which has the effect of modifying the trajectories of the electrons, focusing electrons with specific momenta onto the detector. By converting the measured momentum to energy we are able to fit our data to a linear model based on the Fermi-Kurie plot. We find the value of the kinetic energy of the observed nuclear transition is $T = 0.559 \pm 0.024$ MeV.
\end{abstract}
\end{document}

% Don't print section numbers
\setcounter{secnumdepth}{0}

% the effect of the magnetic field being to cause a cone of electron trajectories diverging from the source to spiral around the axis of the instrument and become a cone converging on to the detector


% The constant $k$ is determined by calibrating the observed spectrum to the well-known K-conversion peak with kinetic energy $T = 624.21$ keV.